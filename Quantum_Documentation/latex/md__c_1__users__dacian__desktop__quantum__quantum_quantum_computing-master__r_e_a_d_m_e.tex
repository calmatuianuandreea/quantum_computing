This project contains libraries written in Java for simulating quantum algorithms. I created this libraries because I saw that there is a lack of libraries for Java that simulate quantum algorithms and the already existing Java based simulators don\textquotesingle{}t expose the A\+PI for using it in new projects.

\subsection*{Getting Started}

You can get a stable version of this project by checking the releases and downloading a zip archive. If you want a development version, you can download the project as a zip or clone it\+:


\begin{DoxyCode}
git clone https://github.com/23ars/quantum\_computing.git
\end{DoxyCode}


\subsubsection*{Prerequisities}

For using this project you\textquotesingle{}ll need a J\+DK ( Java Development Kit) and Apache Maven.

\subsubsection*{Installing}


\begin{DoxyEnumerate}
\item Run {\ttfamily mvn install} in {\bfseries complexnumber} directory.
\item Run {\ttfamily mvn install} in {\bfseries quantum} directory
\item Run {\ttfamily mvn package} in {\bfseries quantumapp} directory.
\end{DoxyEnumerate}

Note\+: If you want to use Eclipse as an I\+DE you need to perform an extra step in each directory\+: {\ttfamily mvn eclipse\+:eclipse}

\subsection*{Running the tests}

\subsubsection*{Complexnumber Junit Tests}

The Junit Tests that are in {\bfseries complexnumber} project verify if operations with complex numbers are performed correctly and give the correct results. For running this tests you will have to run {\ttfamily mvn test} in {\bfseries complexnumber} directory.


\begin{DoxyCode}
@Test
\textcolor{keyword}{public} \textcolor{keywordtype}{void} testConjugate() \{
    ComplexNumber expectedNumber = \textcolor{keyword}{new} ComplexNumber(REAL\_VALUE\_FIRST\_NO, -IMAGINARY\_VALUE\_FIRST\_NO);
    ComplexNumber realNumber = null;
    realNumber = ComplexMath.conjugate(firstNumber);
    assertEquals(expectedNumber, realNumber);
\}
\end{DoxyCode}


\subsubsection*{Quantum\+Lib Junit Tests}

The Junit Tests from {\bfseries quantum} project verify if operations on qubits are performed correctly. To run the tests, you will have to run {\ttfamily mvn test} in {\bfseries quantum} directory.


\begin{DoxyCode}
@RunWith(Suite.class)
@SuiteClasses(\{
    QuantumGatesTest.class,
    QuantumOperationsTest.class,
    QubitTest.class,
    MatrixOperationsTest.class
\})
\textcolor{keyword}{public} \textcolor{keyword}{class} AllTests \{

\}
\end{DoxyCode}


\subsection*{Deployment}

To install the libraries is enough to run {\ttfamily mvn install}. If you want to deploy them, you will have to run {\ttfamily mvn deploy} in each folder.

\subsection*{Built With}


\begin{DoxyItemize}
\item Apache Maven 3.\+3.\+9
\end{DoxyItemize}

\subsection*{Contributing}

Please read \hyperlink{_c_o_n_t_r_i_b_u_t_i_n_g_8md}{C\+O\+N\+T\+R\+I\+B\+U\+T\+I\+NG.md} for details on our code of conduct, and the process for submitting pull requests to us.

\subsection*{Versioning}

We use \href{http://semver.org/}{\tt Sem\+Ver} for versioning. For the versions available, see the \href{https://github.com/23ars/quantum_computing/tags}{\tt tags on this repository}.

\subsection*{Authors}


\begin{DoxyItemize}
\item {\bfseries Mihai Seba} -\/ {\itshape Initial work} -\/ \href{https://github.com/23ars}{\tt 23ars}
\end{DoxyItemize}

See also the list of \href{https://github.com/23ars/quantum_computing/contributors}{\tt contributors} who participated in this project.

\subsection*{License}

This project is licensed under the G\+P\+L-\/3.\+0 License -\/ see the L\+I\+C\+E\+N\+SE.md file for details

\subsection*{Acknowledgments}